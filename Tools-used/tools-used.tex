\chapter{Tools Used}

\paragraph{First, we will provide an overview and analysis of the hardware and software tools employed in the creation of this report, highlighting their functionalities and contributions to the overall research process.}

\section*{Operating Systems}

\vspace{0.5cm}
\includesvg[width=45pt]{./Assets/Linux}
\subsection*{Linux}

Linux was chosen as the operating system for this report's development environment due to its open-source nature, flexibility, and robustness. The Linux ecosystem provided a vast array of tools, libraries, and command-line utilities that greatly facilitated the development and execution of various components. Its command-line interface and shell scripting capabilities enabled efficient automation and customization. The stability, security, and scalability of Linux made it an ideal choice for our report's development environment.

\vspace{1cm}
\includesvg[width=45pt]{./Assets/Windows}
\subsection*{Windows}

Windows operating system was utilized for certain aspects of the report's development and testing. Its user-friendly interface and extensive software compatibility made it accessible to team members with different levels of technical expertise. Windows provided a familiar environment for development, and its compatibility with popular IDEs and tools supported a smooth development workflow. Additionally, Windows-specific software tools and frameworks were leveraged when necessary to ensure compatibility and optimal performance.


\vspace{1cm}
\includesvg[width=45pt]{./Assets/MacOS_logo}
\subsection*{macOS}

macOS served as another operating system employed during the development of this report. Known for its sleek design, stability, and seamless integration with Apple's ecosystem, macOS provided a reliable platform for coding, designing, and testing. Its Unix-based foundation facilitated compatibility with various development tools and command-line utilities. The macOS operating system, along with its intuitive graphical user interface and robust development environment, contributed to an efficient and user-friendly experience for team members working on the report.


\section*{Softwares}


\vspace{0.5cm}
\includesvg[width=45pt]{./Assets/overleaf}
\subsection*{Overleaf} 


Overleaf was utilized as the primary collaborative writing platform for this report. Its robust features and intuitive interface facilitated seamless real-time collaboration among the team members. The platform's LaTeX integration enabled efficient document formatting and enhanced the overall visual appeal of the report.

\vspace{1cm}
\includesvg[width=45pt]{./Assets/LaTeX_logo}
\subsection*{LaTeX}

LaTeX, a typesetting system, played a significant role in the creation of this report. Its powerful document preparation capabilities allowed for precise control over formatting, equations, and typography. LaTeX's extensive collection of packages and templates enhanced the visual appeal and professionalism of the report. With its focus on document structure and separation of content from presentation, LaTeX facilitated efficient collaboration and seamless integration of various sections.

\vspace{1cm}
\includesvg[width=45pt]{./Assets/markdown}
\subsection*{Markdown}

Markdown was utilized for certain sections of the report, particularly in areas where simplicity and readability were prioritized. Markdown's lightweight syntax allowed for quick and straightforward formatting of text, headings, lists, and links. It provided a convenient way to write and edit content without the need for complex markup. Markdown's compatibility with various platforms and its ability to convert to other formats, such as HTML or PDF, made it a versatile choice for creating clear and concise sections within the report.

\vspace{1cm}
\includesvg[width=45pt]{./Assets/git}
\subsection*{Git}

Git played a crucial role in managing the report's source code and version control throughout the project. With its distributed version control system, Git allowed team members to work on different sections of the report simultaneously. Branching and merging capabilities enabled efficient collaboration and seamless integration of changes. Git's commit history and diff functionality provided a detailed overview of code modifications, facilitating easy tracking and reverting of changes when necessary.

\vspace{1cm}
\includesvg[width=45pt]{./Assets/github-2}
\subsection*{GitHub}

GitHub served as the centralized repository and collaboration platform for the report. It offered a host of features that enhanced team coordination and streamlined the development process. The project's repository on GitHub allowed for easy access, sharing, and synchronization of code across team members. GitHub's issue tracking system facilitated efficient communication and task management, ensuring that all team members were aligned with project goals. Additionally, GitHub's pull request mechanism facilitated code review and allowed for iterative improvements to the report's content and codebase.

\vspace{1cm}
\includesvg[width=45pt]{./Assets/docker}
\subsection*{Docker}

Docker was employed to create a standardized and isolated environment for the report's software dependencies. By utilizing Docker containers, the team ensured consistency across different development machines, facilitating smoother deployment and reproducibility. Docker's containerization technology enhanced portability and simplified the setup process for both local development and deployment environments.

\vspace{1cm}
\includesvg[width=45pt]{./Assets/VS_Code}
\subsection*{Visual Studio Code}

Visual Studio Code served as the primary integrated development environment (IDE) for coding and editing tasks related to the report. Its extensive plugin ecosystem and customizable features significantly improved productivity and code quality. The IDE's integrated terminal and version control integration streamlined development workflows and fostered seamless collaboration.

\vspace{1cm}
\includesvg[width=45pt]{./Assets/Node_js}
\subsection*{Node.js}

Node.js, a JavaScript runtime environment, was utilized to execute server-side scripts and handle backend functionalities within the report. Its event-driven architecture and vast package ecosystem allowed for efficient handling of data processing and provided a scalable foundation for building web applications.

\vspace{1cm}
\includesvg[width=45pt]{./Assets/react}
\subsection*{React.js}

React.js, a popular JavaScript library, was employed to develop interactive user interfaces within the report. Its component-based structure facilitated the creation of reusable and modular UI elements, resulting in enhanced code organization and maintainability. React.js's virtual DOM efficiently rendered UI updates, improving the overall performance and user experience.

\vspace{1cm}
\includesvg[width=45pt]{./Assets/vite}
\subsection*{Vite}

Vite, a fast web development build tool, was used to optimize the report's front-end development workflow. Its lightning-fast hot module reloading and built-in bundling capabilities significantly reduced development iteration times. Vite's seamless integration with React.js and its efficient handling of modern JavaScript features contributed to improved development productivity and performance optimization.